\documentclass[11pt,a4paper]{moderncv}

% moderncv themes
\moderncvtheme[blue]{classic}                 % optional argument are 'blue' (default), 'orange', 'red', 'green', 'grey' and 'roman' (for roman fonts, instead of sans serif fonts)
%\moderncvtheme[green]{classic}                % idem

% character encoding
\usepackage[utf8]{inputenc}                   % replace by the encoding you are using

% adjust the page margins
\usepackage[scale=0.76]{geometry}
\setlength{\hintscolumnwidth}{3cm}						% if you want to change the width of the column with the dates
%\AtBeginDocument{\setlength{\maketitlenamewidth}{6cm}}  % only for the classic theme, if you want to change the width of your name placeholder (to leave more space for your address details
%\AtBeginDocument{\recomputelengths}                     % required when changes are made to page layout lengths

% personal data
\firstname{Ing. Ján}
\familyname{Valiska}
\title{Curriculum vitae}               % optional, remove the line if not wanted
\address{Vrbnica 49}{07216}    % optional, remove the line if not wanted
\mobile{+421--949--749--124}                    % optional, remove the line if not wanted
%\phone{phone (optional)}                      % optional, remove the line if not wanted
%\fax{fax (optional)}                          % optional, remove the line if not wanted
%\email{jan.valiska@gmail.com}                      % optional, remove the line if not wanted
\email{jan.valiska@gmail.com}
%\homepage{homepage (optional)}                % optional, remove the line if not wanted
%\extrainfo{additional information (optional)} % optional, remove the line if not wanted
%\photo[64pt]{picture}                         % '64pt' is the height the picture must be resized to and 'picture' is the name of the picture file; optional, remove the line if not wanted
%\quote{Some quote (optional)}                 % optional, remove the line if not wanted

% to show numerical labels in the bibliography; only useful if you make citations in your resume
%\makeatletter
%\makeatother

% bibliography with mutiple entries
%\usepackage{multibib}
%\newcites{book,misc}{{Books},{Others}}

%\nopagenumbers{}                             % uncomment to suppress automatic page numbering for CVs longer than one page
%----------------------------------------------------------------------------------
%            content
%----------------------------------------------------------------------------------
\begin{document}
\maketitle

\section{Vzdelanie}
\cventry{2011--súčasnosť}{Technická univerzita v Košiciach}{Fakulta elektrotechniky a informatiky}{Odbor: Telekomunikácie}{\textit{Doktorandské štúdium}}{}
\cventry{2009--2011}{Technická univerzita v Košiciach}{Fakulta elektrotechniky a informatiky}{Odbor: Multimediálne telekomunikácie}{\textit{Inžinierske štúdium}}{}
\cventry{2005--2009}{Technická univerzita v Košiciach}{Fakulta elektrotechniky a informatiky}{Odbor: Telekomunikácie}{\textit{Bakalárske štúdium}}{}
\cventry{2001--2005}{Stredná priemyselná škola elektrotechnická v Michalovciach}{\newline Odbor: Telekomunikačná technika}{}{}{}

\section{Dizertačná práca}
\cvline{Názov}{\emph{Príspevok k návrhu rýchlych časticových filtrov pre sledovanie objektov vo videosekvenciách}}
\cvline{Školiteľ}{prof.~Ing.~Stanislav Marchevský,~CSc.}
\cvline{Popis}{Práca je zameraná na vývoj časticových filtrov a hodnotenie rýchlosti a presnosti sledovania objektov vo videosekvenciách.}

\section{Pracovné skúsenosti}
\cventry{2012--2015}{T-Systems s.r.o}{ICT Administrátor}{Košice}{}{}{}

\section{Projekty}
\cventry{2013--súčasnosť}{UWB-SeNet}{Short-Range UWB Sensor Networks for Detection, Localization and Tracking of Moving Persons}{\textit{Vývoj a testovanie softvéru UWB radarového systému na 32-bitovej architektúre ARM}}{TUKE Košice}{}{}
\cventry{2013--súčasnosť}{PerLoc-3D-UWB}{Persons Localization in 3D Under Emergency Event based on UWB Radar System}{TUKE Košice}{}{}
\cventry{2011--2012}{CE II}{Rozvoj Centra informačných a komunikačných technológií pre znalostné systémy}{ITMS 26220120030}{TUKE Košice}{}{}

\section{Pedagogické skúsenosti}
\cventry{2013}{SPŠE}{Výučba predmetu \textit{Informatika}}{Programovanie vývojovej dosky Arduino s použitím jazyka C/C++}{Košice}{}{}
\cventry{2012--2014}{T-Systems s.r.o}{Interný školiteľ programovania v jazyku Bash}{Košice}{}{}

\section{Cudzie jazyky}
\cvline{Angličtina}{B1 - mierne pokročilý}

\section{Počitačové zručnosti}
\cvcomputer{Operačné systému}{GNU Linux, MS Windows, Dos, FreeBSD}{Typografia}{\LaTeX, MS Office, OpenOffice\\LibreOffice}
\cvcomputer{Programovanie}{Bash/Zsh, C/C++, Python, Qt, Matlab, Ruby}{Web}{PHP, HTML, Nette Framework, Rails, jQuery, MySQL}
\cvline{OS Linux}{10 ročná prax v používaní a administrácií systému}
\cvline{Vývoj webu}{Tvorenie moderných web aplikácií založených na technológiách ako \textit{PHP Nette} alebo \textit{Ruby on Rails}.}
\cvline{Skriptovanie}{Automatizácia rôznych výpočtových úkonov pomocou Bash/Python skriptov}

\section{Skúsenosti v elektronike}
\cvline{Vložené zariadenia}{Vývoj jednoduchých zariadení pomocou kitu Arduino alebo iných 8-bitových AVR mikroprocesorov. Realizované projekty: Trojmiestný numerický displej s infračerveným riadením, Časovač riadený  rotačným kóderom s 2x16 znakovým LCD displejom, IR riadený HIFI digitálno analógový prevodník(DAC) s použitím Raspberry Pi ARM dosky.}
\cvline{DPS}{Návrh dosiek plošných spojov pomocou softvéru KiCAD. Vytváranie dosiek DPS pomocou metódy využívajúcej negatívny fotorezist.}
\cvline{Simulácie}{Skúsenosť so simuláciou elektronických obvodov pomocou programu ngspice.}

\section{Kompetencie a certifikáty}
\cvlistitem[-]{Vodičský preukaz, typ: B1, B, AM}
\cvlistitem[-]{Cisco Certified Network Assocciate(1-4.semester)}
\cvlistitem[-]{T-Systems internal security certifications}

\section{Záľuby}
\cvline{Open-source softvér}{Mám rád otvorený softvér pre jeho možnosť zmeny a doplnenia zdrojového kódu.}
\cvline{Voľný čas}{Rodina, literatúra, dobrý film a hudba(soft-rock, blues)}

\section{Osobné informácie}
\cvline{Dátum narodenia}{\textit{14.10.1986} Michalovce}
\cvline{Rodinný stav}{ženatý}

%\nocite{*}

\end{document}
